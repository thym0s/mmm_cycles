All (co-)homology groups are understood with rational coefficients.

\subsection{Surface bundles, stratification}
Let $\pi_E : E \rightarrow B$ be an orientable surface bundle
with constant fibre genus g over an orientable closed manifold $B$
of even dimension $2n$, $J$ a fibrewise almost complex structure
compatible with the fibre orientation
and $\omega$ a fibrewise complex cotangent field holomorphic
with respect to $J$.

\begin{defn}
We say that $\omega$ as above is \emph{completely transversal},
if for every point $p \in B$ there is the following data:
\begin{itemize}
  \item An open set $U$ containing $p$,
  \item local coordinates $w=(w_1, \ldots , w_n) : U \rightarrow \mathbb{C}^n$,
        with $w(p) = 0$,
  \item disjoint open subsets $V_1, \ldots, V_k$ of $\pi_E^{-1}(U)$ covering all
        zeroes of $\omega$
  \item fibre coordinates $z_l : V_l \to \mathbb{C}$, such that
        on $V_l$,
        $$\omega = z^{n_l} + w_{n_l-2,l} z^{n_l-2} + \ldots + w_{0,l}$$
        (where $w_{i,j}$ is a renumbering of a subset of the $w_i$).
\end{itemize}
\end{defn}

From now on we assume $\omega$ to be completely transversal.

\noindent \textbf{Notation}.
$\mathcal{E}$ is the $(2g-2)$-fold fibre product of $E$ over $B$. \\
For $I$ a partition, i.e. a quotient, of $\{1, \ldots, 2g-2\}$, let
$\mathcal{E}(I)$ be the subset of $\mathcal{E}$ where $x_i = x_j$ whenever
$i,j$ are identified in $I$. \\
Let $\mathcal{F}$ be the closure $\mathcal{E}$ of the set
$$\{ ( x_1 , \ldots , x_{2g-2} ) \in \mathcal{E} \vert \forall\
i\neq j: x_i \neq x_j \textrm{ and } \omega(x_i) = 0 \} .$$ 

\begin{prop} $\mathcal{F}$ is a closed submanifold of $\mathcal{E}$ that
intersects all $\mathcal{E}(I)$ transversally.
\end{prop}
\begin{proof}
This can be checked in local coordinates: For $z_i$ approaching a
multiple zero of $\omega$, $\mathcal{F}$ is given by the equations
that equal $w_{i,j}$ to the elementary symmetric polynomials in
the corresponding $z_i$, and the sum of them has to equal $0$.
Thus $\mathcal{F}$ is a submanifold by the implicit function theorem.

The second claim also follows from a similar argument.
\end{proof}

We set $\mathcal{F}(I) = \mathcal{F} \cap \mathcal{E}(I)$.
They define the closed strata of a stratification of $\mathcal{F}$.
The canonical projection from $\mathcal{F}$ to $B$ will be denoted
by $\pi$. Note that $\mathcal{F}$ and $B$ have equal dimension.
In a certain sense, $\mathcal{F}$ is a $(2g-2)$-fold branched cover
of $B$, where the branching happens along the $\mathcal{F}(I)$.

\begin{defn}
For a partition $I$, we define the small stabilizer as the group of all
elements $g$ of the symmetric group, such that $g(x) \sim_I x$,
and the big stabilizer via the condition
$x \sim_I y \Leftrightarrow g(x) \sim_I g(y)$.

The order of the small stabilizer is denoted by $e(I)$ and the
order of the big stabilizer is $e(I)f(I)$.
\end{defn}

The big stabilizer is the normalizer of the small stabilizer.

The image of $\mathcal{F}(I)$ under $\pi$ only depends on the
orbit $[I]$ of $I$ under the symmetric group. It will be
denoted by $B([I])$. Note that this might not be a submanifold.
However, the homology class $f(I)[B([I])]$ is well-defined as
the image of $[\mathcal{F}(I)]$.

\subsection{Jet bundles}
The form $\omega$ is a section of the fibrewise cotangent bundle,
which we will denote as $K$. Via fibrewise jet prolongation,
it also defines a section $J_k(\omega)$ in the $k$-fold
fibrewise holomorphic Jet bundle $J_k(K)$, which, as a bundle, is isomorphic to
$K \oplus ( K \otimes K ) \oplus \cdots \oplus ( K \otimes K^{\otimes k} )$.

Let $E_k$ denote the locus of the $k$-fold zeroes of $\omega$ on $E$,
$[E_k]$ its homology class, and $[B_k]$ the image in the homology of $B$.

Note that $[B_k] = [B(\{\{1,\ldots,k\},\{k+1\},\ldots,\{2g-2\}\})]$
in the previous notation.

\begin{prop}
$[E_k]$ is dual to $k! \cdot c_1(K)^k$.
\end{prop}
\begin{proof}
$E_k$ is the vanishing locus of the section $J_{k-1}(\omega)$.
\end{proof}

Hence, $[B_{k+1}]$ is dual to $(k+1)! \cdot \kappa_k$.

\subsection{Wrong way map, intersections}

The wrong-way map $\pi^! : H_\bullet(B) \rightarrow H_\bullet(\mathcal{F})$ can be 
described explicitly on the branching strata:

\begin{prop}
$\pi^!([B([I])]) = f([I]) \sum_{J \in [I]} \mathcal{F}(J) $
\end{prop}

As the dual of the natural homology map, $\pi^!$ respects the intersection
product on homology. Consequently, to compute the intersection products
in the homology of $B$, we can ease the computation by first pulling back
to $\mathcal{F}$. Note that $\pi_\bullet \circ \pi^!$ is multiplication with
$(2g-2)$, so that $\pi^!$ is guaranteed to be injective.

To begin with, we observe the following general fact: If $A,B \subset C \subset D$
are closed (sub-)manifolds, and $A,B$ intersect transverally in $C$,
$$[A] \cap [B] = [C] \cap ( [A] \cap_C [B] ).$$

This formula will allow an inductive computation of the intersections of the
strata $\mathcal{F}(I)$.

We want to compute the intersection of strata of the form $\mathcal{F}(I)$
and $\mathcal{F}(J)$, that means, we want to express it in term of other strata.

\noindent \textbf{ The computation, step 1.}
If $I = \{I_1,\ldots,I_k\}$, then $\mathcal{F}(I)$ is the transversal
intersection of the $\mathcal{F}(\{I_i,\ldots\})$, where only the elements
of $I_i$ are identified. So we can assume that $I$ and $J$
have only one nontrivial element.

\noindent \textbf{ The computation, step 2.}
We assume now that $I=\{I_1,\ldots\}$ and $J=\{J_1,\ldots\}$, where
$I_1$ and $J_1$ are the only equivalence class with more that one element.
Set $I\wedge J = \{ I_1 \cap I_2 , \ldots \}$ and
$I \vee J  = \{ I_1 \cup I_2 , \ldots \}$.
Then
\begin{align}
  [\mathcal{F}(I)] \cap [\mathcal{F}(J)] 
    & = [\mathcal{F}(I \wedge J )] \cap 
       ( [\mathcal{F}(I)] \cap_{\mathcal{F}(I\wedge J)} [\mathcal{F}(J)] ) \\
    & = [\mathcal{F}(I \wedge J )] \cap [\mathcal{F}(I \vee J)] 
\end{align}

\noindent \textbf{ The computation, step 3.}
From now on, we assume in addition that $J$ is a quotient of $I$, so that
$\mathcal{F}(I) \supset \mathcal{F}(J)$. Without loss of generality,
$I_1 = \{ 1 , \ldots , l \}$ and $J_1 = \{ 1 , \ldots , m \}$.
We now want to inductively increase the dimension of $\mathcal{F}(I)$.

So set $\hat{I} = \{ \hat{I}_1 , \ldots \}$
with $\hat{I}_1 = \{ 1 , \ldots , l-1\}$.

The conormal bundle of $\mathcal{F}(I)$ in $\mathcal{F}(\hat{I})$ is
isomorphic to $K$ pulled back over the marked multiple zero.
Over $\mathcal{F}(J)$, the $m$-jet of $\omega$ gives a section of
$K \otimes K^{\otimes m}$, which vanishes exactly at zeroes of higher order.

This means:

\begin{align}
  [\mathcal{F}(I)] \cap [\mathcal{F}(J)]
   & = [\mathcal{F}(\hat{I})] \cap 
      ( [\mathcal{F}(I)] \cap_{\mathcal{F}(\hat{I})} [\mathcal{F}(J)] ) \\
   & = [\mathcal{F}(\hat{I})] \cap
      ( - \frac{1}{m+1} \sum_{i=m+1}^{2g-2}
              \mathcal{F}(\{ J_1 \cup \{ i \} , \ldots \} )
\end{align}

